\lecture{2}{2024-03-01}{First Steps}

\subsection{Primitive Datatypes and Operators}

You have numbers
\begin{lstlisting}[language=Haskell]
3 -- 3
\end{lstlisting}

Math is what you would expect
\begin{lstlisting}[language=Haskell]
1 + 1 -- 2
8 - 1 -- 7
10 * 2 -- 20
35 / 5 -- 7.0
\end{lstlisting}

Division is not integer division by default
\begin{lstlisting}[language=Haskell]
35 / 4 -- 8.75
\end{lstlisting}

integer division
\begin{lstlisting}[language=Haskell]
35 `div` 4 -- 8
\end{lstlisting}

Boolean values are primitives
\begin{lstlisting}[language=Haskell]
True
False
\end{lstlisting}

Boolean operations
\begin{lstlisting}[language=Haskell]
not True -- False
not False -- True
True && False -- False
True || False -- True
1 == 1 -- True
1 /= 1 -- False
1 < 10 -- True
\end{lstlisting}

In the above examples, \textbf{not} is a function that takes one value.
Haskell does not need parentheses for function calls... all the arguments are just
listed after the function. So the general pattern is:

\begin{lstlisting}[language=Haskell]
func arg1 arg2 arg3
\end{lstlisting}

String and characters
\begin{lstlisting}[language=Haskell]
"This is a string."
'a' -- character
'You cant use single quotes for strings.' -- ERROR!
\end{lstlisting}

Strings can be concatenated
\begin{lstlisting}[language=Haskell]
"Hallo " ++ "world!"
\end{lstlisting}

A string is a list of characters
\begin{lstlisting}[language=Haskell]
['H', 'e', 'l', 'l', 'o'] -- "Hello"
\end{lstlisting}

Lists can be indexed with the \textbf{!!} operator followed by an index
\begin{lstlisting}[language=Haskell]
"This is a string" !! 0 -- 'T'
\end{lstlisting}

\subsection{List and Tuples}

Every element in a list must have the same types.
These two lists are equal
\begin{lstlisting}[language=Haskell]
[1, 2, 3, 4, 5]
[1..5]
\end{lstlisting}

Ranges are versatile
\begin{lstlisting}[language=Haskell]
['A'..'F'] -- "ABCDEF"
\end{lstlisting}

You can create a step in a range
\begin{lstlisting}[language=Haskell]
[0,2..10] -- [0, 2, 4, 6, 8, 10]
[5..1] -- [] Haskell defaults to incrementing
[5,4..1] -- [5, 4, 3, 2, 1]
\end{lstlisting}

indexing into a list
\begin{lstlisting}[language=Haskell]
[1..10] !! 3 -- 4 (zero-based indexing)
\end{lstlisting}

You can also have infinite lists in Haskell
\begin{lstlisting}[language=Haskell]
[1..] -- a list of all natural numbers
\end{lstlisting}

Infinite lists work because Haskell has \textit{lazy evaluation}. This means that
Haskell only evaluates things when it needs to. So you can ask for the 1000th element
of your list and Haskell will give it to you.

\begin{lstlisting}[language=Haskell]
[1..] !! 999 -- 1000
\end{lstlisting}

And now Haskell has evaluated elements $1 - 1000$ of this list... but the rest of the
elements of this \textit{infinite} list do not exists yet! Haskell will not
actually evaluate then until it needs to.

Joining two lists
\begin{lstlisting}[language=Haskell]
[1..5] ++ [6..10]
\end{lstlisting}

Adding to the head of a list
\begin{lstlisting}[language=Haskell]
0:[1..5] -- [0, 1, 2, 3, 4, 5]
\end{lstlisting}

Some more list operators
\begin{lstlisting}[language=Haskell]
head [1..5] -- 1
tail [1..5] -- [2, 3, 4, 5]
init [1..5] -- [1, 2, 3, 4]
last [1..5] -- 5
\end{lstlisting}

List comprehensions
\begin{lstlisting}[language=Haskell]
[x*2 | x <- [1..5]] -- [2, 4, 6, 8, 10]
\end{lstlisting}

and with a conditional
\begin{lstlisting}[language=Haskell]
[x*2 | x <- [1..5], x*2 > 4] -- [6, 8, 10]
\end{lstlisting}

Every element in a tuple can be different type, but a tuple has a fiexed length
\\A Tuple:
\begin{lstlisting}[language=Haskell]
("haskell", 1)
\end{lstlisting}

accessing elements of a pair (i.e. a tuple of length 2)
\begin{lstlisting}[language=Haskell]
fst ("haskell", 1) -- "haskell"
snd ("haskell", 1) -- 1
\end{lstlisting}

pair element accessing does not work on n-tuples (i.e. triple, quadruple, etc)
\begin{lstlisting}[language=Haskell]
snd ("snd", "can't tuch this", "da na na na") -- ERROR
\end{lstlisting}

\subsection{Basics of Functions}

A simple function that takes two variables
\begin{lstlisting}[language=Haskell]
add a b = a + b
\end{lstlisting}

Note that if you are using \textbf{ghci} (the Haskell interpreter)
you will need to use \textbf{let}
\begin{lstlisting}[language=Haskell]
let add a b = a + b
\end{lstlisting}

To use a function, use it like this
\begin{lstlisting}[language=Haskell]
add 1 2 -- 3
\end{lstlisting}

You can also put the function name between the two arguments
\begin{lstlisting}[language=Haskell]
1 `add` 2 -- 3
\end{lstlisting}

You can also define functions that have no letters! This lets
you define your own operators! Here's an operator that does
integer division
\begin{lstlisting}[language=Haskell]
(//) a b = a `div` b
35 // 4 -- 8
\end{lstlisting}

Guards: an easy way to do branching in functions
\begin{lstlisting}[language=Haskell]
fib x
  | x < 2 = 1
  | otherwise = fib (x - 1) + fib (x - 2)
\end{lstlisting}

Pattern matching is similar. Here we have given three different
equations that define fib. Haskell will automatically use the first
equation whose left hand side pattern matches the value.
\begin{lstlisting}[language=Haskell]
fib 1 = 1
fib 2 = 2
fib x = fib (x - 1) + fib (x - 2)
\end{lstlisting}

Pattern matching on tuples
\begin{lstlisting}[language=Haskell]
sndOfTriple (_, y, _) = y -- use a wild card (_) to bypass naming unused value
\end{lstlisting}

Pattern matching on lists. Here `x` is the first element
in the list, and `xs` is the rest of the list. We can write
our own map function:
\begin{lstlisting}[language=Haskell]
myMap func [] = []
myMap func (x:xs) = func x:(myMap func xs)
\end{lstlisting}

Anonymous functions are created with a backslash followed by
all the arguments.
\begin{lstlisting}[language=Haskell]
myMap (\x -> x + 2) [1..5] -- [3, 4, 5, 6, 7]
\end{lstlisting}

using fold (called \textit{inject} in some languages) with an anonymous
function. foldl1 means fold left, and use the first value in the
list as the initial value for the accumulator.
\begin{lstlisting}[language=Haskell]
foldl1 (\acc x -> acc + x) [1..5] -- 15
\end{lstlisting}